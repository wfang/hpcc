% -*- LaTeX -*-
%
% This is the master README. Translation to text and HTML files is done
% with HeVeA.
%
% Postprocessing - replace </STYLE> with:
% TT {color:navy}
% H2, H3, H4 {color: #527bbd;}
% H2 {border-bottom: 2px solid silver;}
%.verbatim {background: #ffffee; border: 1px solid silver; padding: 0.5em;}
% </STYLE>

\documentclass[twocolumn]{article}
\usepackage{mathptmx}
\usepackage{url}
\usepackage[margin=2cm]{geometry}
\begin{document}
\title{DARPA/DOE HPC~Challenge Benchmark version 1.4.1}
\author{Piotr Luszczek\footnote{University of Tennessee Knoxville, Innovative
Computing Laboratory}}
\maketitle

\section{Introduction}
This is a suite of benchmarks that measure performance of processor,
memory subsytem, and the interconnect. For details refer to the
HPC~Challenge web site (\url{http://icl.cs.utk.edu/hpcc/}.)

In essence, HPC~Challenge consists of a number of subbenchmarks each
of which tests a different aspect of the system.

If you are familiar with the High Performance Linpack~(HPL) benchmark
code (see the HPL web site:
\texttt{http://www.netlib.org/benchmark/hpl/}) then you can reuse the
build script file~(input for \texttt{make(1)} command) and the input
file that you already have for HPL. The HPC~Challenge benchmark
includes HPL and uses its build script and input files with only
slight modifications. The most important change must be done to the
line that sets the \texttt{TOPdir} variable. For HPC~Challenge, the
variable's value should always be \texttt{../../..} regardless of what
it was in the HPL build script file.

\section{Compiling}
The first step is to create a build script file that reflects
characteristics of your machine. This file is reused by all the
components of the HPC~Challenge suite. The build script file should be
created in the \texttt{hpl} directory. This directory contains
instructions (the files \texttt{README} and \texttt{INSTALL}) on how
to create the build script file for your system. The
\texttt{hpl/setup} directory contains many examples of build script
files. A recommended approach is to copy one of them to the
\texttt{hpl} directory and if it doesn't work then change it.

The build script file has a name that starts with \texttt{Make.}
prefix and usally ends with a suffix that identifies the target
system. For example, if the suffix chosen for the system is
\texttt{Unix}, the file should be named \texttt{Make.Unix}.

To build the benchmark executable (for the system named \texttt{Unix})
type: \texttt{make arch=Unix}.  This command should be run in the top
directory~(not in the \texttt{hpl} directory). It will look in the
\texttt{hpl} directory for the build script file and use it to build
the benchmark executable.

The runtime behavior of the HPC~Challenge source code may be
configured at compiled time by defining a few C preprocessor
symbols. They can be defined by adding appropriate options to
\texttt{CCNOOPT} and \texttt{CCFLAGS} make variables. The former
controls options for source code files that need to be compiled
without aggressive optimizations to ensure accurate generation of
system-specific parameters. The latter applies to the rest of the
files that need good compiler optimization for best performance. To
define a symbol \texttt{S}, the majority of compilers requires option
\texttt{-DS} to be used. Currently, the following options are
available in the HPC~Challenge source code:
\begin{itemize}
\item \texttt{HPCC\_FFT\_235}: if this symbol is defined the FFTE
code (an FFT implementation) will use vector sizes and processor
counts that are not limited to powers of 2. Instead, the vector sizes
and processor counts to be used will be a product of powers of 2, 3,
and 5.
\item \texttt{HPCC\_FFTW\_ESTIMATE}: if this symbol is defined it will
affect the way external FFTW library is called~(it does not have any
effect if the FFTW library is not used). When defined, this symbol
will call the FFTW planning routine with \texttt{FFTW\_ESTIMATE}
flag~(instead of \texttt{FFTW\_MEASURE}). This might result with worse
performance results but shorter execution time of the
benchmark. Defining this symbol may also positively affect the memory
fragmentation caused by the FFTW's planning routine.
\item \texttt{HPCC\_MEMALLCTR}: if this symbol is defined a custom
memory allocator will be used to alleviate effects of memory
fragmentation and allow for larger data sets to be used which may
result in obtaining better performance.
\item \texttt{HPL\_USE\_GETPROCESSTIMES}: if this symbol is defined
then Windows-specific \texttt{GetProcessTimes()} function will be used
to measure the elapsed CPU time.
\item \texttt{USE\_MULTIPLE\_RECV}: if this symbol is defined then multiple non-blocking
receives will be posted simultaneously. By default only one non-blocking
receive is posted.
\item \texttt{RA\_SANDIA\_NOPT}: if this symbol is defined the
HPC~Challenge standard algorithm for Global RandomAccess will not be
used. Instead, an alternative implementation from Sandia
National Laboratory will be used. It routes messages in software
across virtual hyper-cube topology formed from MPI processes.
\item \texttt{RA\_SANDIA\_OPT2}: if this symbol is defined the
HPC~Challenge standard algorithm for Global RandomAccess will not be
used. Instead, instead an alternative implementation from Sandia
National Laboratory will be used. This implementation is optimized for
number of processors being powers of two. The optimizations
are sorting of data before sending and unrolling the data update
loop. If the number of process is not a power two then the code
is the same as the one performed with the \texttt{RA\_SANDIA\_NOPT} setting.
\item \texttt{USING\_FFTW}: if this symbol is defined the standard
HPC~Challenge FFT implemenation~(called FFTE) will not be used.
Instead, FFTW library will be called. Defining the
\texttt{USING\_FFTW} symbol is not sufficient: appropriate flags have
to be added in the make script so that FFTW headers files can be found
at compile time and the FFTW libraries at link time.
\end{itemize}

\section{Runtime Configuration}
The HPC~Challenge is driven by a short input file named
\texttt{hpccinf.txt} that is almost the same as the input file for
HPL~(customarily called \texttt{HPL.dat}). Refer to the directory
\texttt{hpl/www/tuning.html} for details about the input file for
HPL. A sample input file is included with the HPC~Challenge
distribution.

The differences between HPL's input file and HPC~Challenge's input
file can be summarized as follows:

\begin{itemize}
\item Lines 3 and 4 are ignored. The output is always appended to the
file named \texttt{hpccoutf.txt}.
\item There are additional lines~(starting with line 33) that may~(but
do not have to) be used to customize the HPC~Challenge benchmark. They
are described below.
\end{itemize}

The additional lines in the HPC~Challenge input file~(compared to the
HPL input file) are:

\begin{itemize}
\item Lines 33 and 34 describe additional matrix sizes to be used for
running the PTRANS benchmark~(one of the components of the
HPC~Challenge benchmark).
\item Lines 35 and 36 describe additional blocking factors to be used
for running the PTRANS test.
\end{itemize}

Just for completeness, here is the list of lines of the HPC
Challenge's input file and brief description of their meaning:
\begin{itemize}
\item Line 1: ignored
\item Line 2: ignored
\item Line 3: ignored
\item Line 4: ignored
\item Line 5: number of matrix sizes for HPL (and PTRANS)
\item Line 6: matrix sizes for HPL (and PTRANS)
\item Line 7: number of blocking factors for HPL (and PTRANS)
\item Line 8: blocking factors for HPL (and PTRANS)
\item Line 9: type of process ordering for HPL
\item Line 10: number of process grids for HPL (and PTRANS)
\item Line 11: numbers of process rows of each process grid for HPL (and PTRANS)
\item Line 12: numbers of process columns of each process grid for HPL (and PTRANS)
\item Line 13: threshold value not to be exceeded by scaled residual for HPL (and PTRANS)
\item Line 14: number of panel factorization methods for HPL
\item Line 15: panel factorization methods for HPL
\item Line 16: number of recursive stopping criteria for HPL
\item Line 17: recursive stopping criteria for HPL
\item Line 18: number of recursion panel counts for HPL
\item Line 19: recursion panel counts for HPL
\item Line 20: number of recursive panel factorization methods for HPL
\item Line 21: recursive panel factorization methods for HPL
\item Line 22: number of broadcast methods for HPL
\item Line 23: broadcast methods for HPL
\item Line 24: number of look-ahead depths for HPL
\item Line 25: look-ahead depths for HPL
\item Line 26: swap methods for HPL
\item Line 27: swapping threshold for HPL
\item Line 28: form of L1 for HPL
\item Line 29: form of U for HPL
\item Line 30: value that specifies whether equilibration should be used by HPL
\item Line 31: memory alignment for HPL
\item Line 32: ignored
\item Line 33: number of additional problem sizes for PTRANS
\item Line 34: additional problem sizes for PTRANS
\item Line 35: number of additional blocking factors for PTRANS
\item Line 36: additional blocking factors for PTRANS
\end{itemize}

\section{Running}
The exact way to run the HPC~Challenge benchmark depends on the MPI
implementation and system details.  An example command to run the
benchmark could like like this: \texttt{mpirun -np 4 hpcc}. The
meaning of the command's components is as follows:
\begin{itemize}
\item \texttt{mpirun} is the command that starts execution of an MPI
code. Depending on the system, it might also be \texttt{aprun},
\texttt{mpiexec}, \texttt{mprun}, \texttt{poe}, or something
appropriate for your computer.

\item \texttt{-np 4} is the argument that specifies that 4 MPI
processes should be started. The number of MPI processes should be
large enough to accomodate all the process grids specified in the
\texttt{hpccinf.txt} file.

\item \texttt{hpcc} is the name of the HPC~Challenge executable to
run.
\end{itemize}

After the run, a file called \texttt{hpccoutf.txt} is created. It
contains results of the benchmark. This file should be uploaded
through the web form at the HPC~Challenge website.

\section{Source Code Changes across Versions (ChangeLog)}

\subsection{Version 1.4.1 (2010-06-01)}
\begin{enumerate}
\item Added optimized variants of RandomAccess that use Linear Congruential Generator for random number generation.
\item Made corrections to comments that provide definition of the RandomAccess test.
\item Removed initialization of the main array from the timed section of optimized versions of RandomAccess.
\item Updated documentation in README.
\end{enumerate}

\subsection{Version 1.4.0 (2010-03-26)}
\begin{enumerate}
\item Added new variant of RandomAccess that uses Linear Congruential Generator for random number generation.
\item Rearranged the order of benchmarks so that HPL component runs last and may be aborted
if the performance of other components was not satisfactory. RandomAccess is now first to assist in tuning
the code.
\item Added global initialization and finalization routine that allows to properly initialize
and finalize external software and hardware components without changing the rest of the HPCC testing harness.
\item Lack of \texttt{hpccinf.txt} is no longer reported as error but as a warning.
\end{enumerate}

\subsection{Version 1.3.2 (2009-03-24)}
\begin{enumerate}
\item Fixed memory leaks in G-RandomAccess driver routine.
\item Made the check for 32-bit vector sizes in G-FFT optional. MKL allows for 64-bit vector sizes in its FFTW wrapper.
\item Fixed memory bug in single-process FFT.
\item Update documentation (README).
\end{enumerate}

\subsection{Version 1.3.1 (2008-12-09)}
\begin{enumerate}
\item Fixed a dead-lock problem in FFT component due to use of wrong communicator.
\item Fixed the 32-bit random number generator in PTRANS that was using 64-bit
routines from HPL.
\end{enumerate}

\subsection{Version 1.3.0 (2008-11-13)}
\begin{enumerate}
\item Updated HPL component to use HPL 2.0 source code
  \begin{enumerate}
  \item Replaced 32-bit Pseudo Random Number Generator (PRNG) with a 64-bit one.
  \item Removed 3 numerical checks of the solution residual with a single one.
  \item Added support for 64-bit systems with large memory sizes (before they would
  overflow during index calculations 32-bit integers.)
  \end{enumerate}
\item Introduced a limit on FFT vector size so they fit in a 32-bit integer (only
applicable when using FFTW version 2.)
\end{enumerate}

\subsection{Version 1.2.0 (2007-06-25)}

\begin{enumerate}
\item Changes in the FFT component:
   \begin{enumerate}
   \item Added flexibility in choosing vector sizes and processor counts:
   now the code can do powers of 2, 3, and 5 both sequentially and in parallel
   tests.
   \item FFTW can now run with ESTIMATE (not just MEASURE) flag: it might produce
   worse performance results but often reduces time to run the test and cuases
   less memory fragmentation.
   \end{enumerate}
\item Changes in the DGEMM component:
   \begin{enumerate}
   \item Added more comprehensive checking of the numerical properties of the
   test's results.
   \end{enumerate}
\item Changes in the RandomAccess component:
   \begin{enumerate}
   \item Removed time-bound functionality: only runs that perform complete
   computation are now possible.
   \item Made the timing more accurate: main array initialization is not counted
   towards performance timing.
   \item Cleaned up the code: some non-portable C language constructs have been
   removed.
   \item Added new algorithms: new algorithms from Sandia based on hypercube
   network topology can now be chosen at compile time which results on much
   better performance results on many types of parallel systems.
   \item Fixed potential resource leaks by adding function calls rquired by the MPI
   standard.
   \end{enumerate}
\item Changes in the HPL component:
   \begin{enumerate}
   \item Cleaned up reporting of numerics: more accurate printing of scaled
   residual formula.
   \end{enumerate}
\item Changes in the PTRANS component:
   \begin{enumerate}
   \item Added randomization of virtual process grids to measure bandwidth of the
   network more accurately.
   \end{enumerate}
\item Miscellaneous changes:
   \begin{enumerate}
   \item Added better support for Windows-based clusters by taking advantage of
   Win32 API.
   \item Added custom memory allocator to deal with memory fragmentation on some
   systems.
   \item Added better reporting of configuration options in the output file.
   \end{enumerate}
\end{enumerate}

\subsection{Version 1.0.0 (2005-06-11)}

\subsection{Version 0.8beta (2004-10-19)}

\subsection{Version 0.8alpha (2004-10-15)}

\subsection{Version 0.6beta (2004-08-21)}

\subsection{Version 0.6alpha (2004-05-31)}

\subsection{Version 0.5beta (2003-12-01)}

\subsection{Version 0.4alpha (2003-11-13)}

\subsection{Version 0.3alpha (2004-11-05)}

\end{document}
